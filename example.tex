\documentclass{slideshow}

\title{How to Create Elegant \LaTeX{ }Slideshows}
\subtitle{a tutorial on the \texttt{slideshow} document class}
\author
{
  Matthew Nazari \\
  \textit{Department of Computer Science} \\
  \textit{Harvard University} \\
  Cambridge, MA \\
  \href{mailto:matthewnazari@college.harvard.edu}{matthewnazari@college.harvard.edu}
  \and
  Cole French \\
  \textit{Department of Computer Science} \\
  \textit{Harvard University} \\
  Cambridge, MA \\
  \href{mailto:cfrench@college.harvard.edu}{cfrench@college.harvard.edu}
}
\date{Spring 2022}
\conference{Harvard University}

\begin{document}
  \slide[Title Slide]
  The title slide is automatically created. \par
  It supports
  \begin{itemize}
    \item \verb|\title{}|
    \item \verb|\subtitle{}|
    \item \verb|\author{}|
    \item \verb|\conference{}|
    \item \verb|\date{}|
  \end{itemize}

  \headerslide[Header Slides]{Create a Header Slide With\par {\normalfont\Large\texttt{\textbackslash headerslide[optional PDF bookmark name]\{Slide Title\}}}}

  \slide[Slides]
  You create a new slide using \verb|\slide[Slide Title]|. \par
  This will create PDF bookmark with the name of the slide title.

  \slide[Subslides]
  This is slide \theslide.
  \label{subslide}

  \slide
  This is slide \theslide. \par
  Notice that it is a subslide of \autoref{subslide}. \par
  You create a subslide using \verb|\slide|, by not passing a slide title argument.
\end{document}
